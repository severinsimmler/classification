%%%%%%%%%%%%%%%%%%%%%%%%%%%%%%%%%%%%%%%%%%%%%%%%%%%%%%%%%%%%%%%%%%%%%%%%
\chapter{Einführung}
\label{einfuehrung}
%%%%%%%%%%%%%%%%%%%%%%%%%%%%%%%%%%%%%%%%%%%%%%%%%%%%%%%%%%%%%%%%%%%%%%%%

ABC


%%%%%%%%%%%%%%%%%%%%%%%%%%%%%%%%%%%%%%%%%%%%%%%%%%%%%%%%%%%%%%%%%%%%%%%%
\chapter{Theoretische Grundlagen}
\label{grundlagen}
%%%%%%%%%%%%%%%%%%%%%%%%%%%%%%%%%%%%%%%%%%%%%%%%%%%%%%%%%%%%%%%%%%%%%%%

ABC

%%%%%%%%%%%%%%%%%%%%%%%%%%%%%%%%%%%%%%%%%%%%%%%%%%%%%%%%%%%%%%%%%%%%%%%%
\section{Klassifikation}
\label{korpora}
%%%%%%%%%%%%%%%%%%%%%%%%%%%%%%%%%%%%%%%%%%%%%%%%%%%%%%%%%%%%%%%%%%%%%%%%

ABC


%%%%%%%%%%%%%%%%%%%%%%%%%%%%%%%%%%%%%%%%%%%%%%%%%%%%%%%%%%%%%%%%%%%%%%%%
\section{Machine Learning}
\label{korpora}
%%%%%%%%%%%%%%%%%%%%%%%%%%%%%%%%%%%%%%%%%%%%%%%%%%%%%%%%%%%%%%%%%%%%%%%%

ABC


%%%%%%%%%%%%%%%%%%%%%%%%%%%%%%%%%%%%%%%%%%%%%%%%%%%%%%%%%%%%%%%%%%%%%%%%
\section{Deep Learning}
\label{korpora}
%%%%%%%%%%%%%%%%%%%%%%%%%%%%%%%%%%%%%%%%%%%%%%%%%%%%%%%%%%%%%%%%%%%%%%%%

ABC


%%%%%%%%%%%%%%%%%%%%%%%%%%%%%%%%%%%%%%%%%%%%%%%%%%%%%%%%%%%%%%%%%%%%%%%%
\chapter{Experimente}
\label{experimente}
%%%%%%%%%%%%%%%%%%%%%%%%%%%%%%%%%%%%%%%%%%%%%%%%%%%%%%%%%%%%%%%%%%%%%%%%

ABC

%%%%%%%%%%%%%%%%%%%%%%%%%%%%%%%%%%%%%%%%%%%%%%%%%%%%%%%%%%%%%%%%%%%%%%%%
\section{Korpora}
\label{korpora}
%%%%%%%%%%%%%%%%%%%%%%%%%%%%%%%%%%%%%%%%%%%%%%%%%%%%%%%%%%%%%%%%%%%%%%%%

ABC

\begin{table}
\centering
\begin{tabular}{lllll}
\toprule
 &    Dramen &   Romane &  Wikipedia & Zeitung \\
\midrule
Fragmente       &     781 &     781 &        781 &      781 \\
Tokens     &   84711 &   69864 &      91422 &    93263 \\
Types      &   19867 &    8724 &      25972 &    26857 \\
Minimum    &      20 &      19 &         35 &       50 \\
Maximum    &     144 &     290 &        150 &      204 \\
Durchschnitt &  108 &   89 &     117 &   119 \\
Standardabweichung &    16 &   38 &      20 &    44 \\
Klassenverteilung &  304:296:181 &                                  304:296:181 &  304:296:181 &  304:296:181 \\
\bottomrule
\end{tabular}
\caption{%
Verwendete Korpora. Dramen, Wikipedia und Zeitung wurden auf die Größe des Roman Korpus downgesampled, um einen vergleichbaren Ausgangspunkt annähernd zu gewährleisten. 
}
\label{korpusstats}
\end{table}


%%%%%%%%%%%%%%%%%%%%%%%%%%%%%%%%%%%%%%%%%%%%%%%%%%%%%%%%%%%%%%%%%%%%%%%%
\section{Versuchsaufbau}
\label{versuchsaufbau}
%%%%%%%%%%%%%%%%%%%%%%%%%%%%%%%%%%%%%%%%%%%%%%%%%%%%%%%%%%%%%%%%%%%%%%%%




%%%%%%%%%%%%%%%%%%%%%%%%%%%%%%%%%%%%%%%%%%%%%%%%%%%%%%%%%%%%%%%%%%%%%%%%
\section{Evaluation}
\label{evaluation}
%%%%%%%%%%%%%%%%%%%%%%%%%%%%%%%%%%%%%%%%%%%%%%%%%%%%%%%%%%%%%%%%%%%%%%%%

\begin{table}
\centering
\begin{tabular}{lrrrr}
\toprule
{} & Dramen  &    Romane &  Wikipedia &  Zeitung\\
\midrule

Random&  .3544&.3544&.3544&.3544 \\
Logistic Regression&.6709&.6203&.9241&.7975 \\
Support Vector Machine&\textbf{.7215}&\textbf{.7089}&.9114&.7848 \\
BERT (feature-based)&.6962&.6329&\textbf{.9367}&.8101 \\
BERT (fine-tuned)&.6709&.6203&.9241&\textbf{.9494} \\

\bottomrule
\end{tabular}
\caption{TODO!}
\end{table}





\begin{figure}
\centering
\begin{subfigure}[b]{.45\linewidth}
\includesvg[width=\linewidth]{/home/severin/Downloads/plots/logistic-regression-dramen.svg}
\caption{Logistic Regression}\label{fig:drama-log}
\end{subfigure}
\begin{subfigure}[b]{.45\linewidth}
\includesvg[width=\linewidth]{/home/severin/Downloads/plots/svm-dramen.svg}
\caption{Support Vector Machine}\label{fig:drama-svm}
\end{subfigure}

\begin{subfigure}[b]{.45\linewidth}
\includesvg[width=\linewidth]{/home/severin/Downloads/plots/feature-based-dramen.svg}
\caption{BERT (feature-based)}\label{fig:drama-feat}
\end{subfigure}
\begin{subfigure}[b]{.45\linewidth}
\includesvg[width=\linewidth]{/home/severin/Downloads/plots/fine-tuned-dramen.svg}
\caption{BERT (fine-tuned)}\label{fig:drama-fine}
\end{subfigure}
\caption{Normalisierte Konfusionsmatrizen der verschiedenen Ansätze für das Dramen Korpus; je dunkler ein Feld, desto mehr Fragmente wurden der Klasse in der Spalte bzw. Reihe zugeordnet. Support Vector Machine (b) schneidet hier durchschnittlich am besten ab.}
\label{fig:drama}
\end{figure}




%%%%%%%%%%%%%%%%%%%%%%%%%%%%%


\begin{figure}
\centering
\begin{subfigure}[b]{.45\linewidth}
\includesvg[width=\linewidth]{/home/severin/Downloads/plots/logistic-regression-romane.svg}
\caption{Logistic Regression}\label{fig:romane-log}
\end{subfigure}
\begin{subfigure}[b]{.45\linewidth}
\includesvg[width=\linewidth]{/home/severin/Downloads/plots/svm-romane.svg}
\caption{Support Vector Machine}\label{fig:romane-svm}
\end{subfigure}

\begin{subfigure}[b]{.45\linewidth}
\includesvg[width=\linewidth]{/home/severin/Downloads/plots/feature-based-romane.svg}
\caption{BERT (feature-based)}\label{fig:romane-feat}
\end{subfigure}
\begin{subfigure}[b]{.45\linewidth}
\includesvg[width=\linewidth]{/home/severin/Downloads/plots/fine-tuned-romane.svg}
\caption{BERT (fine-tuned)}\label{fig:romane-fine}
\end{subfigure}
\caption{Normalisierte Konfusionsmatrizen der verschiedenen Ansätze für das Roman Korpus; je dunkler ein Feld, desto mehr Fragmente wurden der Klasse in der Spalte bzw. Reihe zugeordnet. Support Vector Machine (b) schneidet hier durchschnittlich am besten ab.}
\label{fig:roman}
\end{figure}




%%%%%%%%%%%%%%%%%%%%%%%%%%%%%%%%%%%%%%%%%%%%%%%%%%%%%%%%%
\begin{figure}
\centering
\begin{subfigure}[b]{.45\linewidth}
\includesvg[width=\linewidth]{/home/severin/Downloads/plots/logistic-regression-wikipedia.svg}
\caption{Logistic Regression}\label{fig:wikipedia-log}
\end{subfigure}
\begin{subfigure}[b]{.45\linewidth}
\includesvg[width=\linewidth]{/home/severin/Downloads/plots/svm-wikipedia.svg}
\caption{Support Vector Machine}\label{fig:wikipedia-svm}
\end{subfigure}

\begin{subfigure}[b]{.45\linewidth}
\includesvg[width=\linewidth]{/home/severin/Downloads/plots/feature-based-wikipedia.svg}
\caption{BERT (feature-based)}\label{fig:wikipedia-feat}
\end{subfigure}
\begin{subfigure}[b]{.45\linewidth}
\includesvg[width=\linewidth]{/home/severin/Downloads/plots/fine-tuned-wikipedia.svg}
\caption{BERT (fine-tuned)}\label{fig:wikipedia-fine}
\end{subfigure}
\caption{Normalisierte Konfusionsmatrizen der verschiedenen Ansätze für das Wikipedia Korpus; je dunkler ein Feld, desto mehr Fragmente wurden der Klasse in der Spalte bzw. Reihe zugeordnet. Feature-based BERT (c) schneidet hier durchschnittlich am besten ab.}
\label{fig:roman}
\end{figure}






%%%%%%%%%%%%%%%%%%%%%%%%%%%%%%%%%%%%%%%%%%%%%%%%%%
\begin{figure}
\centering
\begin{subfigure}[b]{.45\linewidth}
\includesvg[width=\linewidth]{/home/severin/Downloads/plots/logistic-regression-zeitung.svg}
\caption{Logistic Regression}\label{fig:zeitung-log}
\end{subfigure}
\begin{subfigure}[b]{.45\linewidth}
\includesvg[width=\linewidth]{/home/severin/Downloads/plots/svm-zeitung.svg}
\caption{Support Vector Machine}\label{fig:zeitung-svm}
\end{subfigure}

\begin{subfigure}[b]{.45\linewidth}
\includesvg[width=\linewidth]{/home/severin/Downloads/plots/feature-based-zeitung.svg}
\caption{BERT (feature-based)}\label{fig:zeitung-feat}
\end{subfigure}
\begin{subfigure}[b]{.45\linewidth}
\includesvg[width=\linewidth]{/home/severin/Downloads/plots/fine-tuned-zeitung.svg}
\caption{BERT (fine-tuned)}\label{fig:zeitung-fine}
\end{subfigure}
\caption{Normalisierte Konfusionsmatrizen der verschiedenen Ansätze für das Zeitungs Korpus; je dunkler ein Feld, desto mehr Fragmente wurden der Klasse in der Spalte bzw. Reihe zugeordnet. Fine-tuned BERT (d) schneidet hier durchschnittlich am besten ab.}
\label{fig:roman}
\end{figure}





%%%%%%%%%%%%%%%%%%%%%%%%%%%%%%%%%%%%%%%%%%%%%%%%%%%%%%%%%%%%%%%%%%%%%%%%
\chapter{Diskussion}
\label{diskussion}
%%%%%%%%%%%%%%%%%%%%%%%%%%%%%%%%%%%%%%%%%%%%%%%%%%%%%%%%%%%%%%%%%%%%%%%%



%%%%%%%%%%%%%%%%%%%%%%%%%%%%%%%%%%%%%%%%%%%%%%%%%%%%%%%%%%%%%%%%%%%%%%%%
\chapter{Ausblick und Fazit}
\label{ausblick}
%%%%%%%%%%%%%%%%%%%%%%%%%%%%%%%%%%%%%%%%%%%%%%%%%%%%%%%%%%%%%%%%%%%%%%%%



\nocite{*}
