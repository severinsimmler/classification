\documentclass{mimosis}

\usepackage{metalogo}
\usepackage{svg}


\usepackage{geometry}
 \geometry{
 a4paper,
 top=25mm,
 bottom=20mm,
 right=40mm,
 left=30mm,
 footskip=9mm
 }
\AtBeginDocument{\fontsize{12pt}{16pt}\selectfont}

\usepackage{pdfpages}


\usepackage{algorithmicx, algpseudocode}
\usepackage[Algorithm,ruled]{algorithm}
\usepackage{tikz}
\usetikzlibrary{matrix}


\makeatletter
\renewcommand*{\ps@plain}{%
  \let\@mkboth\@gobbletwo
  \let\@oddhead\@empty
  \def\@oddfoot{%
    \reset@font
    \hfil
    \thepage
    % \hfil % removed for aligning to the right
  }%
  \let\@evenhead\@empty
  \let\@evenfoot\@oddfoot
}
\makeatother
\pagestyle{plain}
%%%%%%%%%%%%%%%%%%%%%%%%%%%%%%%%%%%%%%%%%%%%%%%%%%%%%%%%%%%%%%%%%%%%%%%%
% Some of my favourite personal adjustments
%%%%%%%%%%%%%%%%%%%%%%%%%%%%%%%%%%%%%%%%%%%%%%%%%%%%%%%%%%%%%%%%%%%%%%%%
%
% These are the adjustments that I consider necessary for typesetting
% a nice thesis. However, they are *not* included in the template, as
% I do not want to force you to use them.

% This ensures that I am able to typeset bold font in table while still aligning the numbers
% correctly.
\usepackage{etoolbox}

\usepackage[binary-units=true]{siunitx}
\DeclareSIUnit\px{px}

\definecolor{link-gray}{HTML}{575757}

\sisetup{%
  detect-all           = true,
  detect-family        = true,
  detect-mode          = true,
  detect-shape         = true,
  detect-weight        = true,
  detect-inline-weight = math,
}

%%%%%%%%%%%%%%%%%%%%%%%%%%%%%%%%%%%%%%%%%%%%%%%%%%%%%%%%%%%%%%%%%%%%%%%%
% Hyperlinks & bookmarks
%%%%%%%%%%%%%%%%%%%%%%%%%%%%%%%%%%%%%%%%%%%%%%%%%%%%%%%%%%%%%%%%%%%%%%%%

\usepackage[%
  colorlinks = true,
  citecolor  = link-gray,
  linkcolor  = link-gray,
  urlcolor   = link-gray,
  ]{hyperref}

\usepackage{bookmark}

%%%%%%%%%%%%%%%%%%%%%%%%%%%%%%%%%%%%%%%%%%%%%%%%%%%%%%%%%%%%%%%%%%%%%%%%
% Bibliography
%%%%%%%%%%%%%%%%%%%%%%%%%%%%%%%%%%%%%%%%%%%%%%%%%%%%%%%%%%%%%%%%%%%%%%%%
%
% I like the bibliography to be extremely plain, showing only a numeric
% identifier and citing everything in simple brackets. The first names,
% if present, will be initialized. DOIs and URLs will be preserved.

\usepackage[%
  autocite     = plain,
  backend      = biber,
  doi          = true,
  url          = true,
  giveninits   = true,
  hyperref     = true,
  maxbibnames  = 9,
  maxcitenames = 2,
  sortcites    = true,
  style        = authoryear-icomp
  ]{biblatex}

%%%%%%%%%%%%%%%%%%%%%%%%%%%%%%%%%%%%%%%%%%%%%%%%%%%%%%%%%%%%%%%%%%%%%%%%
% Some adjustments to make the bibliography more clean
%%%%%%%%%%%%%%%%%%%%%%%%%%%%%%%%%%%%%%%%%%%%%%%%%%%%%%%%%%%%%%%%%%%%%%%%
%
% The subsequent commands do the following:
%  - Removing the month field from the bibliography
%  - Fixing the Oxford commma
%  - Suppress the "in" for journal articles
%  - Remove the parentheses of the year in an article
%  - Delimit volume and issue of an article by a colon ":" instead of
%    a dot ""
%  - Use commas to separate the location of publishers from their name
%  - Remove the abbreviation for technical reports
%  - Display the label of bibliographic entries without brackets in the
%    bibliography
%  - Ensure that DOIs are followed by a non-breakable space
%  - Use hair spaces between initials of authors
%  - Make the font size of citations smaller
%  - Fixing ordinal numbers (1st, 2nd, 3rd, and so) on by using
%    superscripts

% Remove the month field from the bibliography. It does not serve a good
% purpose, I guess. And often, it cannot be used because the journals
% have some crazy issue policies.
\AtEveryBibitem{\clearfield{month}}
\AtEveryCitekey{\clearfield{month}}

% Fixing the Oxford comma. Not sure whether this is the proper solution.
% More information is available under [1] and [2].
%
% [1] http://tex.stackexchange.com/questions/97712/biblatex-apa-style-is-missing-a-comma-in-the-references-why
% [2] http://tex.stackexchange.com/questions/44048/use-et-al-in-biblatex-custom-style
%
\AtBeginBibliography{%
  \renewcommand*{\finalnamedelim}{%
    \ifthenelse{\value{listcount} > 2}{%
      \addcomma
      \addspace
      \bibstring{and}%
    }{%
      \addspace
      \bibstring{and}%
    }
  }
}

% Suppress "in" for journal articles. This is unnecessary in my opinion
% because the journal title is typeset in italics anyway.
\renewbibmacro{in:}{%
  \ifentrytype{article}
  {%
  }%
  % else
  {%
    \printtext{\bibstring{in}\intitlepunct}%
  }%
}

% Remove the parentheses for the year in an article. This removes a lot
% of undesired parentheses in the bibliography, thereby improving the
% readability. Moreover, it makes the look of the bibliography more
% consistent.
\renewbibmacro*{issue+date}{%
  \setunit{\addcomma\space}
    \iffieldundef{issue}
      {\usebibmacro{date}}
      {\printfield{issue}%
       \setunit*{\addspace}%
       \usebibmacro{date}}%
  \newunit}

% Delimit the volume and the number of an article by a colon instead of
% by a dot, which I consider to be more readable.
\renewbibmacro*{volume+number+eid}{%
  \printfield{volume}%
  \setunit*{\addcolon}%
  \printfield{number}%
  \setunit{\addcomma\space}%
  \printfield{eid}%
}

% Do not use a colon for the publisher location. Instead, connect
% publisher, location, and date via commas.
\renewbibmacro*{publisher+location+date}{%
  \printlist{publisher}%
  \setunit*{\addcomma\space}%
  \printlist{location}%
  \setunit*{\addcomma\space}%
  \usebibmacro{date}%
  \newunit%
}

% Ditto for other entry types.
\renewbibmacro*{organization+location+date}{%
  \printlist{location}%
  \setunit*{\addcomma\space}%
  \printlist{organization}%
  \setunit*{\addcomma\space}%
  \usebibmacro{date}%
  \newunit%
}

% Do not abbreviate "technical report".
\DefineBibliographyStrings{english}{%
  bibliography = {Bibliography},
  techreport   = {technical report},
}

% Display the label of a bibliographic entry in bare style, without any
% brackets. I like this more than the default.
%
% Note that this is *really* the proper and official way of doing this.
\DeclareFieldFormat{labelnumberwidth}{#1\adddot}

% Ensure that DOIs are followed by a non-breakable space.
\DeclareFieldFormat{doi}{%
  \mkbibacro{DOI}\addcolon\addnbspace
    \ifhyperref
      {\href{http://dx.doi.org/#1}{\nolinkurl{#1}}}
      %
      {\nolinkurl{#1}}
}

% Use proper hair spaces between initials as suggested by Bringhurst and
% others.
\renewcommand*\bibinitdelim {\addnbthinspace}
\renewcommand*\bibnamedelima{\addnbthinspace}
\renewcommand*\bibnamedelimb{\addnbthinspace}
\renewcommand*\bibnamedelimi{\addnbthinspace}

% Make the font size of citations smaller. Depending on your selected
% font, you might not need this.
\renewcommand*{\citesetup}{%
  \biburlsetup
  \small
}

\DeclareLanguageMapping{british}{bibliography-correct-ordinals}
\DeclareLanguageMapping{english}{bibliography-correct-ordinals}

\bibliography{Thesis}

%%%%%%%%%%%%%%%%%%%%%%%%%%%%%%%%%%%%%%%%%%%%%%%%%%%%%%%%%%%%%%%%%%%%%%%%
% Fonts
%%%%%%%%%%%%%%%%%%%%%%%%%%%%%%%%%%%%%%%%%%%%%%%%%%%%%%%%%%%%%%%%%%%%%%%%

\ifxetexorluatex
  \setmainfont{Minion Pro}
\else
  \usepackage[lf]{ebgaramond}
  \newcommand{\fakebf}{\fontfamily{mdugm}\fontseries{b}\selectfont}
  \DeclareTextFontCommand{\textbf}{\fakebf}
  \usepackage[oldstyle,scale=0.7]{sourcecodepro}
  \singlespacing
\fi

\renewcommand{\th}{\textsuperscript{\textup{th}}\xspace}

\newacronym{NER}{NER}{Named Entity Recognition}
\newacronym{biLSTM}{biLSTM}{Bidirektionales Long Short-Term Memory Netzwerk}
\newacronym{DROC}{DROC}{Deutsches Roman Korpus}
\newacronym{RNN}{RNN}{Rekurrentes neuronales Netzwerk}
\newacronym{TP}{TP}{True Positive}
\newacronym{FP}{FP}{False Positive}
\newacronym{FN}{FN}{False Negative}
\newacronym{AppTdfW}{AppTdfW}{Appelativ Teil der fiktionalen Welt}
\newacronym{AppA}{AppA}{Appelativ Abstraktum}



%\newglossaryentry{LaTeX}{%
%  name        = {\LaTeX},
%  description = {A document preparation system},
%  sort        = {LaTeX},
%}

%\newglossaryentry{Real numbers}{%
%  name        = {$\real$},
%  description = {The set of real numbers},
%  sort        = {Real numbers},
%}

\makeindex
\makeglossaries
\makeindex

%%%%%%%%%%%%%%%%%%%%%%%%%%%%%%%%%%%%%%%%%%%%%%%%%%%%%%%%%%%%%%%%%%%%%%%%
% Incipit
%%%%%%%%%%%%%%%%%%%%%%%%%%%%%%%%%%%%%%%%%%%%%%%%%%%%%%%%%%%%%%%%%%%%%%%%

\title{Das Problem begrenzter Daten}

\begin{document}

\frontmatter
  \begin{titlepage}
  \vspace*{5cm}
  \makeatletter
  \begin{center}
    \begin{Huge}
      \@title
    \end{Huge}\\[0.1cm]
    \vspace{10mm}
    Seminararbeit im Rahmen der Veranstaltung\\
    Praxis digitaler Objekte 2: Einführung in Deep Learning\\
    im Sommersemester 2019\\
    an der\\
    \textsc{Julius-Maximilians-Universität Würzburg\\
            Institut für Deutsche Philologie\\
            Lehrstuhl für Computerphilologie und\\
            Neuere Deutsche Literaturgeschichte}\\
  \end{center}
  \vfill
  \begin{flushleft}
    Severin Simmler\\
    \href{mailto:severin.simmler@stud-mail.uni-wuerzburg.de}{severin.simmler@stud-mail.uni-wuerzburg.de}\\
    Matrikelnummer: 2028090\\
    Studienfach: Digital Humanities, M.A.\\
    Fachsemester: 4
  \end{flushleft}
  \vspace{3mm}
  \begin{flushleft}
    betreut von\\
    Prof. Dr. Fotis Jannidis
  \end{flushleft}
\end{titlepage}

\newpage
\null
\thispagestyle{empty}
\newpage

  %\begin{center}
  \textsc{Zusammenfassung}
\end{center}
%
\noindent
%
Thema der vorliegenden Arbeit ist, inwiefern sogenannte \textit{word embeddings} dazu beitragen, die 

  \begin{center}
  \textsc{Selbstständigkeitserklärung}
\end{center}

\noindent Hiermit erkläre ich, dass ich die vorliegende Arbeit selbstständig verfasst und keine anderen als die angegebenen Hilfsmittel verwendet habe. Insbesondere versichere ich, dass ich alle wörtlichen und sinngemäßen Übernahmen aus anderen Werken als solche kenntlich gemacht habe.
\vspace{3cm}

\begin{tabular}{p{10mm}>{\centering\arraybackslash}p{50mm}p{10mm}
>{\centering\arraybackslash}p{50mm}p{10mm}}
&\textit{\large Würzburg,}&&& \\
&\textit{\large den 30. September 2019}&&\hrulefill& \\
&\small Ort, Datum&&\small Severin Simmler&
\end{tabular}

  
  
  \selectlanguage{ngerman}
  \tableofcontents
  \listoffigures
  \listoftables

\mainmatter

  %%%%%%%%%%%%%%%%%%%%%%%%%%%%%%%%%%%%%%%%%%%%%%%%%%%%%%%%%%%%%%%%%%%%%%%%
\chapter{Einführung}
\label{einfuehrung}
%%%%%%%%%%%%%%%%%%%%%%%%%%%%%%%%%%%%%%%%%%%%%%%%%%%%%%%%%%%%%%%%%%%%%%%%

ABC


%%%%%%%%%%%%%%%%%%%%%%%%%%%%%%%%%%%%%%%%%%%%%%%%%%%%%%%%%%%%%%%%%%%%%%%%
\chapter{Theoretische Grundlagen}
\label{grundlagen}
%%%%%%%%%%%%%%%%%%%%%%%%%%%%%%%%%%%%%%%%%%%%%%%%%%%%%%%%%%%%%%%%%%%%%%%

ABC

%%%%%%%%%%%%%%%%%%%%%%%%%%%%%%%%%%%%%%%%%%%%%%%%%%%%%%%%%%%%%%%%%%%%%%%%
\section{Klassifikation}
\label{korpora}
%%%%%%%%%%%%%%%%%%%%%%%%%%%%%%%%%%%%%%%%%%%%%%%%%%%%%%%%%%%%%%%%%%%%%%%%

ABC


%%%%%%%%%%%%%%%%%%%%%%%%%%%%%%%%%%%%%%%%%%%%%%%%%%%%%%%%%%%%%%%%%%%%%%%%
\section{Machine Learning}
\label{korpora}
%%%%%%%%%%%%%%%%%%%%%%%%%%%%%%%%%%%%%%%%%%%%%%%%%%%%%%%%%%%%%%%%%%%%%%%%

ABC


%%%%%%%%%%%%%%%%%%%%%%%%%%%%%%%%%%%%%%%%%%%%%%%%%%%%%%%%%%%%%%%%%%%%%%%%
\section{Deep Learning}
\label{korpora}
%%%%%%%%%%%%%%%%%%%%%%%%%%%%%%%%%%%%%%%%%%%%%%%%%%%%%%%%%%%%%%%%%%%%%%%%

ABC


%%%%%%%%%%%%%%%%%%%%%%%%%%%%%%%%%%%%%%%%%%%%%%%%%%%%%%%%%%%%%%%%%%%%%%%%
\chapter{Experimente}
\label{experimente}
%%%%%%%%%%%%%%%%%%%%%%%%%%%%%%%%%%%%%%%%%%%%%%%%%%%%%%%%%%%%%%%%%%%%%%%%

ABC

%%%%%%%%%%%%%%%%%%%%%%%%%%%%%%%%%%%%%%%%%%%%%%%%%%%%%%%%%%%%%%%%%%%%%%%%
\section{Korpora}
\label{korpora}
%%%%%%%%%%%%%%%%%%%%%%%%%%%%%%%%%%%%%%%%%%%%%%%%%%%%%%%%%%%%%%%%%%%%%%%%

ABC

\begin{table}
\centering
\begin{tabular}{lllll}
\toprule
 &    Dramen &   Romane &  Wikipedia & Zeitung \\
\midrule
Fragmente       &     781 &     781 &        781 &      781 \\
Tokens     &   84711 &   69864 &      91422 &    93263 \\
Types      &   19867 &    8724 &      25972 &    26857 \\
Minimum    &      20 &      19 &         35 &       50 \\
Maximum    &     144 &     290 &        150 &      204 \\
Durchschnitt &  108 &   89 &     117 &   119 \\
Standardabweichung &    16 &   38 &      20 &    44 \\
Klassenverteilung &  304:296:181 &                                  304:296:181 &  304:296:181 &  304:296:181 \\
\bottomrule
\end{tabular}
\caption{%
Verwendete Korpora. Dramen, Wikipedia und Zeitung wurden auf die Größe des Roman Korpus downgesampled, um einen vergleichbaren Ausgangspunkt annähernd zu gewährleisten. 
}
\label{korpusstats}
\end{table}


%%%%%%%%%%%%%%%%%%%%%%%%%%%%%%%%%%%%%%%%%%%%%%%%%%%%%%%%%%%%%%%%%%%%%%%%
\section{Versuchsaufbau}
\label{versuchsaufbau}
%%%%%%%%%%%%%%%%%%%%%%%%%%%%%%%%%%%%%%%%%%%%%%%%%%%%%%%%%%%%%%%%%%%%%%%%




%%%%%%%%%%%%%%%%%%%%%%%%%%%%%%%%%%%%%%%%%%%%%%%%%%%%%%%%%%%%%%%%%%%%%%%%
\section{Evaluation}
\label{evaluation}
%%%%%%%%%%%%%%%%%%%%%%%%%%%%%%%%%%%%%%%%%%%%%%%%%%%%%%%%%%%%%%%%%%%%%%%%

\begin{table}
\centering
\begin{tabular}{lrrrr}
\toprule
{} & Dramen  &    Romane &  Wikipedia &  Zeitung\\
\midrule

Random&  .3544&.3544&.3544&.3544 \\
Logistic Regression&.6709&.6203&.9241&.7975 \\
Support Vector Machine&\textbf{.7215}&\textbf{.7089}&.9114&.7848 \\
BERT (feature-based)&.6962&.6329&\textbf{.9367}&.8101 \\
BERT (fine-tuned)&.6709&.6203&.9241&\textbf{.9494} \\

\bottomrule
\end{tabular}
\caption{TODO!}
\end{table}





\begin{figure}
\centering
\begin{subfigure}[b]{.45\linewidth}
\includesvg[width=\linewidth]{/home/severin/Downloads/plots/logistic-regression-dramen.svg}
\caption{Logistic Regression}\label{fig:drama-log}
\end{subfigure}
\begin{subfigure}[b]{.45\linewidth}
\includesvg[width=\linewidth]{/home/severin/Downloads/plots/svm-dramen.svg}
\caption{Support Vector Machine}\label{fig:drama-svm}
\end{subfigure}

\begin{subfigure}[b]{.45\linewidth}
\includesvg[width=\linewidth]{/home/severin/Downloads/plots/feature-based-dramen.svg}
\caption{BERT (feature-based)}\label{fig:drama-feat}
\end{subfigure}
\begin{subfigure}[b]{.45\linewidth}
\includesvg[width=\linewidth]{/home/severin/Downloads/plots/fine-tuned-dramen.svg}
\caption{BERT (fine-tuned)}\label{fig:drama-fine}
\end{subfigure}
\caption{Normalisierte Konfusionsmatrizen der verschiedenen Ansätze für das Dramen Korpus; je dunkler ein Feld, desto mehr Fragmente wurden der Klasse in der Spalte bzw. Reihe zugeordnet. Support Vector Machine (b) schneidet hier durchschnittlich am besten ab.}
\label{fig:drama}
\end{figure}




%%%%%%%%%%%%%%%%%%%%%%%%%%%%%


\begin{figure}
\centering
\begin{subfigure}[b]{.45\linewidth}
\includesvg[width=\linewidth]{/home/severin/Downloads/plots/logistic-regression-romane.svg}
\caption{Logistic Regression}\label{fig:romane-log}
\end{subfigure}
\begin{subfigure}[b]{.45\linewidth}
\includesvg[width=\linewidth]{/home/severin/Downloads/plots/svm-romane.svg}
\caption{Support Vector Machine}\label{fig:romane-svm}
\end{subfigure}

\begin{subfigure}[b]{.45\linewidth}
\includesvg[width=\linewidth]{/home/severin/Downloads/plots/feature-based-romane.svg}
\caption{BERT (feature-based)}\label{fig:romane-feat}
\end{subfigure}
\begin{subfigure}[b]{.45\linewidth}
\includesvg[width=\linewidth]{/home/severin/Downloads/plots/fine-tuned-romane.svg}
\caption{BERT (fine-tuned)}\label{fig:romane-fine}
\end{subfigure}
\caption{Normalisierte Konfusionsmatrizen der verschiedenen Ansätze für das Roman Korpus; je dunkler ein Feld, desto mehr Fragmente wurden der Klasse in der Spalte bzw. Reihe zugeordnet. Support Vector Machine (b) schneidet hier durchschnittlich am besten ab.}
\label{fig:roman}
\end{figure}




%%%%%%%%%%%%%%%%%%%%%%%%%%%%%%%%%%%%%%%%%%%%%%%%%%%%%%%%%
\begin{figure}
\centering
\begin{subfigure}[b]{.45\linewidth}
\includesvg[width=\linewidth]{/home/severin/Downloads/plots/logistic-regression-wikipedia.svg}
\caption{Logistic Regression}\label{fig:wikipedia-log}
\end{subfigure}
\begin{subfigure}[b]{.45\linewidth}
\includesvg[width=\linewidth]{/home/severin/Downloads/plots/svm-wikipedia.svg}
\caption{Support Vector Machine}\label{fig:wikipedia-svm}
\end{subfigure}

\begin{subfigure}[b]{.45\linewidth}
\includesvg[width=\linewidth]{/home/severin/Downloads/plots/feature-based-wikipedia.svg}
\caption{BERT (feature-based)}\label{fig:wikipedia-feat}
\end{subfigure}
\begin{subfigure}[b]{.45\linewidth}
\includesvg[width=\linewidth]{/home/severin/Downloads/plots/fine-tuned-wikipedia.svg}
\caption{BERT (fine-tuned)}\label{fig:wikipedia-fine}
\end{subfigure}
\caption{Normalisierte Konfusionsmatrizen der verschiedenen Ansätze für das Wikipedia Korpus; je dunkler ein Feld, desto mehr Fragmente wurden der Klasse in der Spalte bzw. Reihe zugeordnet. Feature-based BERT (c) schneidet hier durchschnittlich am besten ab.}
\label{fig:roman}
\end{figure}






%%%%%%%%%%%%%%%%%%%%%%%%%%%%%%%%%%%%%%%%%%%%%%%%%%
\begin{figure}
\centering
\begin{subfigure}[b]{.45\linewidth}
\includesvg[width=\linewidth]{/home/severin/Downloads/plots/logistic-regression-zeitung.svg}
\caption{Logistic Regression}\label{fig:zeitung-log}
\end{subfigure}
\begin{subfigure}[b]{.45\linewidth}
\includesvg[width=\linewidth]{/home/severin/Downloads/plots/svm-zeitung.svg}
\caption{Support Vector Machine}\label{fig:zeitung-svm}
\end{subfigure}

\begin{subfigure}[b]{.45\linewidth}
\includesvg[width=\linewidth]{/home/severin/Downloads/plots/feature-based-zeitung.svg}
\caption{BERT (feature-based)}\label{fig:zeitung-feat}
\end{subfigure}
\begin{subfigure}[b]{.45\linewidth}
\includesvg[width=\linewidth]{/home/severin/Downloads/plots/fine-tuned-zeitung.svg}
\caption{BERT (fine-tuned)}\label{fig:zeitung-fine}
\end{subfigure}
\caption{Normalisierte Konfusionsmatrizen der verschiedenen Ansätze für das Zeitungs Korpus; je dunkler ein Feld, desto mehr Fragmente wurden der Klasse in der Spalte bzw. Reihe zugeordnet. Fine-tuned BERT (d) schneidet hier durchschnittlich am besten ab.}
\label{fig:roman}
\end{figure}





%%%%%%%%%%%%%%%%%%%%%%%%%%%%%%%%%%%%%%%%%%%%%%%%%%%%%%%%%%%%%%%%%%%%%%%%
\chapter{Diskussion}
\label{diskussion}
%%%%%%%%%%%%%%%%%%%%%%%%%%%%%%%%%%%%%%%%%%%%%%%%%%%%%%%%%%%%%%%%%%%%%%%%



%%%%%%%%%%%%%%%%%%%%%%%%%%%%%%%%%%%%%%%%%%%%%%%%%%%%%%%%%%%%%%%%%%%%%%%%
\chapter{Ausblick und Fazit}
\label{ausblick}
%%%%%%%%%%%%%%%%%%%%%%%%%%%%%%%%%%%%%%%%%%%%%%%%%%%%%%%%%%%%%%%%%%%%%%%%



\nocite{*}


% This ensures that the subsequent sections are being included as root
% items in the bookmark structure of your PDF reader.
\bookmarksetup{startatroot}
\backmatter

  \begingroup
    \let\clearpage\relax
    \glsaddall
    \printglossary[type=\acronymtype]
    \newpage
    \printglossary
  \endgroup

  \printindex
  \printbibliography

\end{document}
